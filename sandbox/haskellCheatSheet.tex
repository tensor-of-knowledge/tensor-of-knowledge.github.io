\documentclass{article}
\usepackage[utf8]{inputenc}
\usepackage[russian]{babel}
\usepackage{cmap}
\usepackage{amsfonts,amsmath}
\usepackage{geometry}
\usepackage{listings}
\usepackage{minted}
\geometry{verbose,a5paper,tmargin=0.8cm,bmargin=1.5cm,lmargin=0.2cm,rmargin=0.2cm}
\pdfcompresslevel=9

%VARS%
%! { 
%! "TITLE": "Шпаргалка по haskell",
%! "DESCRIPTION": "Основные приемы программирования на haskell",
%! "AUTHOR": "edmi",
%! "TAGS": "haskell"
%! }
%VARS%

% WARNING!!!
% this file uses minted package, make sure it is installed
% sudo apt-get install texlive-*
% sudo apt-get install python-pygments

\begin{document}

    \huge{\textbf{Шпаргалка по haskell}}\\\small
    
    \large{Замечания}\small\\
    
    -- Списки можно сравнивать, списки \textbf{большей} длины \textbf{больше}:
    \begin{minted}{haskell}
        [1,2,3] > [100,200]
    \end{minted}
    
    -- Оператор $..$ в генераторах:
    \begin{minted}{haskell}
        [a..b] = [a,a+1,...b], b>=a
        [a..b] = [], b<a
    \end{minted}
    
    -- $let$ в генетаторах:
    \begin{minted}{haskell}
        [x*y | x<-[1,2,3], let y=2] == [2,4,6]
    \end{minted}
    
    -- Числа с плавающей точкой лучше \textbf{не использовать в интервалах!} Могут вызвать непредсказуемое поведение.
    
    -- Порядок применения функций в генераторах:
    \begin{minted}{haskell}
        [x `f` y | x<-[a1,b1,...], y<-[a2,b2,...]] == 
        == [a1 `f` a2, a1 `f` b2, ..., b1 `f` a2]
    \end{minted}
    
    -- Можно объявлять инфиксную функцию:
    \begin{minted}{haskell}
        b `a` c = b + c
        5 `a` 10 == 15
    \end{minted}
    
    -- Вычитание в сечениях:
    \begin{minted}{haskell}
        map (-4) [4,4,4] == BOTTOM
        map (subtract 4) [4,4,4] == [0,0,0]    
    \end{minted}

    -- Каррирование в лямбда-выражениях:
    \begin{minted}{haskell}
        \x y -> x+y == \x -> \y -> x+y
    \end{minted}
    
    -- Операторы $(\$)$ и $(.)$ \textbf{правоассоциативны}
    
    -- Именованные образцы:
    \begin{minted}{haskell}
        (f name@(x:xs) = ...) == (f (x:xs) = ... where name=(x:xs))
    \end{minted}
    
    \large{Импорт модулей}\small\\
    
    -- Все функции в глобальное пространство имен:
    \begin{minted}{haskell}
        import A.B
    \end{minted}
    
    -- Только функции $a, b$:
    \begin{minted}{haskell}
        import A.B (a, b)
    \end{minted}
    
    -- Кроме $a, b$:
    \begin{minted}{haskell}
        import A.B hiding (a, b)
    \end{minted}
    
    -- Вызов через $A.B.func$:
    \begin{minted}{haskell}
        import qualified A.B
    \end{minted}    
    
    -- Вызов через $C.func$:
    \begin{minted}{haskell}
        import qualified A.B as C
    \end{minted}        
    
    \large{Классы типов}\small\\    

    \begin{minted}{haskell}
        name :: (Class x) => T1 [->T2->...->TN], Ti=x
    \end{minted}
    
    \textbf{Eq} -- равны (=) либо не равны (/=)
    
    \textbf{Ord} -- отношения порядка (>, <, >=, <=)
    
    \textbf{Show} -- применима функция show:
    \begin{minted}{haskell}
        show :: (Class x) => x -> String
    \end{minted}    
    
    \textbf{Read} -- применима функция read :: x:
    \begin{minted}{haskell}
        read :: (Class x) => String -> x
    \end{minted}
    
    \textbf{Enum} -- значения можно пронумеровать
    
    \textbf{Bounded} -- есть верхняя/нижняя граница
    
    \textbf{Num} -- экземпляры ведут себя как числа
    
    \textbf{Floating} -- как числа с плавающей точкой
    
    \textbf{Integral} -- как целые числа
    
    
    
    
 
\end{document}


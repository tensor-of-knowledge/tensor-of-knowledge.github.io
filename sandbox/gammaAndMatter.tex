\documentclass{article}
\usepackage[utf8]{inputenc}
\usepackage[russian]{babel}
\usepackage{cmap}
\usepackage{amsfonts,amsmath}
\usepackage{geometry}
\usepackage{listings}
\geometry{verbose,a5paper,tmargin=0.8cm,bmargin=1.5cm,lmargin=0.2cm,rmargin=0.2cm}
\pdfcompresslevel=9

%VARS%
%! { 
%! "TITLE": "Взаимодействие гамма-квантов с веществом",
%! "DESCRIPTION": "Кратко о взаимодействии гамма-квантов с веществом",
%! "AUTHOR": "edmi",
%! "TAGS": "физика, ядерная физика"
%! }
%VARS%

\begin{document}

    \huge{\textbf{Взаимодействие гамма-квантов с веществом}}\\\small
    
    При взаимодействии гамма-квантов с веществом на малых энергиях наибольший вклад вносит фотоэффект.
    
    \begin{equation}
        h\nu = A + \frac{mv^2}{2}
    \end{equation}
    
    На средних энергиях -- комптоновский эффект.
    
    \begin{equation}
        \nu' = \nu\frac{1}{1+\frac{h\nu}{m_e c^2}(1-\cos\theta)}
    \end{equation}
    
    На больших энергиях -- рождение электрон-позитронной пары.
    
\end{document}


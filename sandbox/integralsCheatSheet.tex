\documentclass{article}
\usepackage[utf8]{inputenc}
\usepackage[russian]{babel}
\usepackage{cmap}
\usepackage{amsfonts,amsmath}
\usepackage{geometry}
\usepackage{listings}
\geometry{verbose,a5paper,tmargin=0.8cm,bmargin=1.5cm,lmargin=0.2cm,rmargin=0.2cm}
\pdfcompresslevel=9

%VARS%
%! { 
%! "TITLE": "Шпаргалка по интегралам",
%! "DESCRIPTION": "Основные методы интегрирования и таблица интегралов",
%! "AUTHOR": "edmi",
%! "TAGS": "математический анализ"
%! }
%VARS%

\begin{document}

    \huge{\textbf{Шпаргалка по интегралам}}\\\small
    
    \large{Таблица интегралов}\small
    
    \begin{equation}
        \int a^x dx = \frac{a^x}{\ln a} + C
    \end{equation}
    
    \begin{equation}
        \int \frac{dx}{\sqrt{a^2-x^2}} = \arcsin\frac{x}{a} + C
    \end{equation}
    
    \begin{equation}
        \int \frac{dx}{a^2 + x^2} = \frac{1}{a}\arctg\frac{x}{a} + C
    \end{equation}
    
    \begin{equation}
        \int \frac{dx}{a^2 - x^2} = \frac{1}{2a}\ln |\frac{a+x}{a-x}| + C
    \end{equation}
    
    \begin{equation}
        \int \frac{dx}{x^2\pm a^2} = \ln |x+\sqrt{x^2\pm a^2}| + C
    \end{equation}\\
    
    \large{Интегрирование дробно-рациональной функции}\small
    
    1) Определить, является ли дробь правильной (по степеням многочленов). Если нет -- поделить числитель на знаменатель.
    
    2) Разложить знаменатель на множители.
    
    3) Представляем дробь в виде суммы рациональных дробей. Знаменатели -- множители из п. 2, числители -- выражения с новыми переменными $A, B, C, ...$, причем:
    
    -- Если есть кратные множители $x^n$, требуется написать в сумму $n$ дробей вида $\frac{A_i}{x^i},\;\;i=1..n$

    -- Если среди множителей есть неразложимые квадратные трехчлены, в соотв. числителе -- выражения вида $Ax+B$.
    
    -- Если знаменатель -- $x^n$ или $(x+\alpha)^n$, то в соотв. числителе -- единственная переменная.

    4) Привести полученную сумму к общему знаменателю и приравнять к исходной дроби
    
    5) Найти значения новых переменных методом неопределенных коэффициентов (сгруппировать слева и справа эти переменные при $x$ в одной степени, составить и решить систему линейных уравнений)

    6) Интегрировать найденную в п. 4 сумму с найденными в п. 5 числителями\\
    
    \large{Подстановки Эйлера}\small   
    
    Используются для интегрирования рациональных функций, содержащих $\sqrt{ax^2+bx+c}$.
    
    Первая подстановка, если $a>0$:
    \begin{equation}
        \sqrt{ax^2+bx+c} = \pm t \pm \sqrt{a} x
    \end{equation}
    
    Вторая подстановка, если $c>0$:
    \begin{equation}
        \sqrt{ax^2 + bx + c} = \pm xt \pm \sqrt{x}
    \end{equation}
    
    Третья подстановка, если подкоренное выражение имеет два действительных корня:
    \begin{equation}
        \sqrt{ax^2+bx+c}=t(x-\lambda), \lambda \text{-- один из корней}
    \end{equation}
\end{document}


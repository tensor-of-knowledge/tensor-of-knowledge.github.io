\documentclass{article}
\usepackage[utf8]{inputenc}
\usepackage[russian]{babel}
\usepackage{cmap}
\usepackage{amsfonts,amsmath}
\usepackage{geometry}
\usepackage{listings}
\geometry{verbose,a5paper,tmargin=0.8cm,bmargin=1.5cm,lmargin=0.2cm,rmargin=0.2cm}
\pdfcompresslevel=9

%VARS%
%! { 
%! "TITLE": "Формальная постановка задачи машинного обучения",
%! "DESCRIPTION": "Общие сведения о машинном обучении",
%! "AUTHOR": "edmi",
%! "TAGS": "машинное обучение"
%! }
%VARS%

\begin{document}

    \huge{\textbf{Формальная постановка задачи машинного обучения}}\\\small
    
    Машинное обучение -- "наука о том, как восстановить функцию по точкам".\\
    
    \textbf{Пусть:}
    
    $X$ -- множество объектов
    
    $Y$ -- множество ответов
    
    $y: X\rightarrow Y$ -- неизвестная зависимость (target function)\\
    
    \textbf{Дано:}
    
    $\{x_1,...,x_l\}\subset X$ -- обучающая выборка (training sample)
    
    $y_i = y(x_i),\;i=1,...,l$ -- известные ответы\\
    
    \textbf{Найти:}
    
    $a: X\rightarrow Y$ -- алгоритм, решающую функцию (decision function), приближающую $y$ на всем множестве $X$.\\
    
    \textbf{Признаковое описание объектов}
    
    $f_j: X\rightarrow D_j,\;j=1,...,n$ -- признаки объектов (features)\\
    
    $D_j = \{0,1\}$ -- бинарный признак $f_j$
    
    $|D_j| < \infty$ -- номинальный признак $f_j$
    
    $|D_j| < \infty,\;D_j$ упорядочено -- порядковый признак $f_j$
    
    $D_j = \mathbb{R}$ -- количественный признак $f_j$\\
    
    \textit{Представление обучающей выборки -- матрица "объекты-признаки" (номера столбцов -- номера признаков, номера строк -- номера объектов)}\\

    \textbf{Разновидности ответов -- типы задач}
    
    $Y = \{-1,+1\}$ -- задача \textbf{классификации} (classification) на 2 класса
    
    $Y = \{1,...,M\}$ -- задача \textbf{классификации} на $M$ непересекающихся классов
    
    $Y = \{0,1\}^M$ -- задача \textbf{классификации} на $M$ классов, которые могут пересекаться
    
    $Y = \mathbb{R}$ или $Y = \mathbb{R}^m$ -- задача \textbf{восстановления регрессии} (regression)

    $Y$ -- конечное упорядоченное множество -- задача \textbf{ранжирования} (ranking)\\
    
    \textbf{Предсказательная модель}
    
    Модель (predictive model) -- параметрическое семейство функций:
    \begin{equation}
        A = \{a(x)=g(x,\theta)\; |\; \theta\in\Theta\}
    \end{equation}
    
    \textbf{Функционалы качества}
    
    Один из методов решения задач машинного обучения -- сведение их к задачам оптимизации, т.е. выбор оптимального вектора $\theta$ через максимизацию точности предсказываемых ответов. Эту точность показывают функционалы качества.\\
    
    $\mathcal{L}(a,x)$ -- функция потерь (loss function) -- величина ошибки алгоритма $a\in A$ на объекте $x\in X$.
    
    Для задач классификации:
    
    $\mathcal{L}(a,x) = [a(x)\neq y(x)]$ -- индикатор ошибки\\
    
    Для задач регрессии:
    
    $\mathcal{L}(a,x) = |a(x)-y(x)|$ -- абсолютное значение ошибки
    
    $\mathcal{L}(a,x) = (a(x)-y(x))^2$ -- квадратичная ошибка\\
    
    Эмпирический риск -- функционал качества алгоритма $a$ на $X^l$
    
    \begin{equation}
        Q(a,X^l) = \frac{1}{l}\sum_{i=1}^l \mathcal(a,x_i)
    \end{equation}
    
    Решение задачи -- минимизация эмпирического риска (empirical risk minimization):
    
    \begin{equation}
        \mu(X^l) = \arg \min_{a\in A} Q(a,X^l)
    \end{equation}
    
    \textbf{Переобучение}\\
    
    -- это когда найденный алгоритм хорошо работает на обучающей выборке и плохо на тестовой.\\
    
    Эмпирические оценки переобучения:\\
    
    $HO(\mu,X^l,X^k)=Q(\mu(X^l),X^k)$ -- Эмпирический риск на тестовых данных (hold-out)
    
    $LOO(\mu,X^l) = \frac{1}{L}\sum_{i=1}^L \mathcal{L}(\mu(X^L\setminus \{x_i\}),x_i),\;\;L=l+1$ -- Скользящий контроль (leave-one-out)
    
    $CV(\mu,X^L)=\frac{1}{N}\sum_{n=1}^N Q(\mu(X_n^l),X_n^k),\;\;X^L = X_n^l \sqcup X_n^k,\;\; L = l+k$ -- Кросс-проверка (cross-validation) по $N$ разбиениям\\
    
    Выбор $\mu$, для которого такая оценка минимальна, может снять состояние переобучения.
    
\end{document}

